\documentclass[12pt]{article}

%% preamble: Keep it clean; only include those you need
\usepackage{amsmath}
\usepackage[margin = 1in]{geometry}
\usepackage{graphicx}
\usepackage{booktabs}
\usepackage{natbib}

%% for double spacing
\usepackage{setspace}

% for space filling
\usepackage{lipsum}
% highlighting hyper links
\usepackage[colorlinks=true, citecolor=blue]{hyperref}


%% meta data

\title{Review Report for\\
  ``The Title of The Paper Being Reviewed''}

\begin{document}
\maketitle


This paper presents online updating approach for robust regression in
big streamed data setting. The authors proposed a modified Huber's loss
function which retain the asymptotic properties of the resulting
estimator. The point estimator results are insensitive to the tuning
parameter of the modification. Further, the approach was built into
LASSO model selection with promising results. The manuscript is
well-written. Nonetheless, the paper seems to left a few properties of
importance unstudied or unreported as outlined in the following.

\section*{General Comments}

\begin{enumerate}
\item
  The variance of the renewable estimator was derived and its estimator
  was not explicitly presented. For statistical inferences, point
  estimator is not enough; one would need a variance estimator of the
  renewable estimator. Can the variance be estimated in the renewable
  algorithm setting like other online-updating approaches? If so, the
  performance of the estimator needs to be reported in the simulation
  studies.
\item
  The algorithms requires \(\tau\) as an input, which depends on the
  unknown standard deviation of the regression error. Therefore, the
  algorithms are not practically applicable unless \(\tau\) or
  \(\sigma\) is given. To estimate \(\sigma\), there are different
  choices available and a rubost estimator is probably preferred. The
  estimation of \(\sigma\) and the robustness of the results to the
  estimation method need to be studied.
\item
  The variable selection procedure with a batch-specific \(\lambda\)
  appears somewhat unnatural to me. If the sample size in each batch is
  not big enough, an important variable with a coefficient of smaller
  magnitude may never get selected. The authors used
  \(\beta_0 = (1, 1, 2, 3, 4, 5, 0, \ldots, 0)\), where all coefficients
  are large and easy to be selected. What if some coefficients are 0.5
  as in the simulation study reported in Tibrishani's orignal work?
\item
  I would propose a renewable solution path in the model selection. That
  is, we start with a grid of \(\lambda\) values; at each batch, we
  update the solution path indexed by the lambda values. At the end, you
  get a solution path for the whole data and an optimal \(\lambda\)
  value could selected by certain criteria. This way, important
  variables with small coefficients might be identified as more batches
  become available.
\item
  In the study for the model comparison, only MSEs for the estimated
  regression coeffiients were compared. It would be informative to also
  report performance measures on model selection such as false negatives
  and false positives.
\item
  The real application could be strengthened. First, the problem
  description is not detailed enough. A general statistical reader would
  not understand what timbre variable and timbre covariance variables
  mean; nor why they can predict the year of a song. Sedond, the
  application in the UCI repository has been studied extensively in the
  literature but the analysis does not report any comparison with
  existing literatures. In general, enough details need to be given for
  readers to reproduce the analysis (e.g., how was the threshold
  chosen?).
\end{enumerate}

\section*{Specific Comments}

\begin{enumerate}
\item
  Page 3: \(\sigma\) has not been defined yet. It should be defined when
  introducing \(\epsilon\). Further, the 95\% relative efficieny was
  meant for normally distributed regression errors instead of any error.
\item
  Figure 2 caption: range instead of rang.
\item
  Page 8: \(\mathbf{R}\) should have a subscript to indicate that it
  depends on the sample size. Same for the remainder term in
  Equation\textasciitilde(3.4).
\item
  Equation (2.10): The threshold 10 needs some justification.
\item
  Table 1: \(p\) and p are different notations. Same for \(N(0,1)\) and
  N(0,1).
\item
  Tables: four digits after the decimal places are distracting. Three is
  enough. For entries that are too small, consider reporting them scaled
  by 100, for example.
\end{enumerate}
\end{document}